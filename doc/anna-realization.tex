\documentclass[a4paper]{article}
\usepackage[utf8x]{inputenc}
\usepackage[croatian]{babel}
\usepackage{geometry}
\usepackage{tikz}
\usepackage{subfigure}
\usepackage{todonotes}

\usetikzlibrary{shapes}

\begin{document}
  \title{ANNA}
  \maketitle

  \section{Kostur}
  Kostur sačinjavaju sučelja koja se nalaze pod
  hr.anna.interfaces. Sučelja predstavljaju očekivanja i mogućnosti
  svake od komponenata.

  Protokoli koje kostur omogućava su neovisni o implementaciji. Za
  primjer potpunosti uzmimo trojku IBusMaster, IBus i
  IBusUnit. IBusMaster može pristupati jedinicama na sabirnici preko
  sljedećeg protokola:

  IBusMaster poziva IBus.requestWrite i preko nje postavlja zahtjev za
  čitanje. Sabirnica obavlja translaciju adrese i dodatne operacije
  potrebne za pristup i prosljeđuje zahtjev preko
  IBusUnit.requestWrite. Jedinica na sabirnici nakon obavljene
  operacije uzvraća preko IBus.busUnitWriteCallback. Sabirnica opet
  obavlja određene translacije i na kraju dostavlja informaciju preko
  IBusMaster.busWriteCallback. Protokol je ilustriran na slici
  \ref{bus_protocol}.

  \subsection{Kritike i poboljšanja}
  Protokol je većinom prilagođen za komunikacije na sabirnicama gdje
  postoji jedan ili par upravljača (mastera) i više jedinica. Pozitivna
  stvar ovoga je što je većina internih protokola master-slave koncepta,
  a i neki eksterni protokoli. Ova sučelja trenutno ne predviđaju složene
  arbitracije i to je jedno od područja koje se može dodatno razvijati.

  Sučelja također omogućuju bridgeve među sabirnicama. Bridge jedinica
  je za jednu sabirnicu samo IBusUnit, a za drugu IBusMaster.

  \todo{Dodati tablicu koja nudi summary protokola koje je moguće/nije
    moguće implementirati}

  \tikzstyle{busmaster} = [draw=red, fill=blue!20, thick, rectangle,
  rounded corners, inner sep=0.2cm]

\tikzstyle{busunit} = [draw=red, fill=blue!20, thick,
  rectangle, rounded corners, inner sep=0.2cm]

\tikzstyle{bus} = [draw=black, fill=white, thick, rectangle, minimum
  width=5cm, minimum height=1pt]

\begin{figure}[hbt!]
  \begin{center}
    \subfigure[Prije početka komunikacije]{
      \begin{tikzpicture}
        \node [bus]       (bus)                    {IBus};
        \node [busmaster] (busmaster) at (-4.0, 0) {IBusMaster};
        \node [busunit]   (busunit1)  at (-1.5, 1) {IBusUnit};
        \node [busunit]   (busunit2)  at ( 1.5, 1) {IBusUnit};
        \node [busunit]   (busunit3)  at (-1.5,-1) {IBusUnit};
        \node [busunit]   (busunit4)  at ( 1.5,-1) {IBusUnit};

        \draw[<->,thick] (busmaster.east) -- (bus.west);
        \draw[<->,thick] (busunit1.south) -- (busunit1.south |- bus.north);
        \draw[<->,thick] (busunit2.south) -- (busunit2.south |- bus.north);
        \draw[<->,thick] (busunit3.north) -- (busunit3.north |- bus.south);
        \draw[<->,thick] (busunit4.north) -- (busunit4.north |- bus.south);
      \end{tikzpicture}
    }
    \subfigure[IBus.requestBusWrite(IBusMaster,Address,Data)]{
      \begin{tikzpicture}
        \node [bus]       (bus)                    {IBus};
        \node [busmaster] (busmaster) at (-4.0, 0) {IBusMaster};
        \node [busunit]   (busunit1)  at (-1.5, 1) {IBusUnit};
        \node [busunit]   (busunit2)  at ( 1.5, 1) {IBusUnit};
        \node [busunit]   (busunit3)  at (-1.5,-1) {IBusUnit};
        \node [busunit]   (busunit4)  at ( 1.5,-1) {IBusUnit};

        \draw[->,ultra thick,color=red] (busmaster.east) -- (bus.west);
        \draw[<->,thick] (busunit1.south) -- (busunit1.south |- bus.north);
        \draw[<->,thick] (busunit2.south) -- (busunit2.south |- bus.north);
        \draw[<->,thick] (busunit3.north) -- (busunit3.north |- bus.south);
        \draw[<->,thick] (busunit4.north) -- (busunit4.north |- bus.south);
      \end{tikzpicture}
    }
    \subfigure[IBusUnit.requestBusUnitWrite(IBus,TranslatedAddress,Data)]{
      \begin{tikzpicture}
        \node [bus]       (bus)                    {IBus};
        \node [busmaster] (busmaster) at (-4.0, 0) {IBusMaster};
        \node [busunit]   (busunit1)  at (-1.5, 1) {IBusUnit};
        \node [busunit]   (busunit2)  at ( 1.5, 1) {IBusUnit};
        \node [busunit]   (busunit3)  at (-1.5,-1) {IBusUnit};
        \node [busunit]   (busunit4)  at ( 1.5,-1) {IBusUnit};

        \draw[<->,thick] (busmaster.east) -- (bus.west);
        \draw[<-,ultra thick, color=red] (busunit1.south) --
        (busunit1.south |- bus.north);
        \draw[<->,thick] (busunit2.south) -- (busunit2.south |- bus.north);
        \draw[<->,thick] (busunit3.north) -- (busunit3.north |- bus.south);
        \draw[<->,thick] (busunit4.north) -- (busunit4.north |- bus.south);
      \end{tikzpicture}
    }
    \subfigure[IBus.busUnitWriteCallback(IBusUnit,TranslatedAddress)]{
      \begin{tikzpicture}
        \node [bus]       (bus)                    {IBus};
        \node [busmaster] (busmaster) at (-4.0, 0) {IBusMaster};
        \node [busunit]   (busunit1)  at (-1.5, 1) {IBusUnit};
        \node [busunit]   (busunit2)  at ( 1.5, 1) {IBusUnit};
        \node [busunit]   (busunit3)  at (-1.5,-1) {IBusUnit};
        \node [busunit]   (busunit4)  at ( 1.5,-1) {IBusUnit};

        \draw[<->,thick] (busmaster.east) -- (bus.west);
        \draw[->,ultra thick, color=red] (busunit1.south) --
        (busunit1.south |- bus.north);
        \draw[<->,thick] (busunit2.south) -- (busunit2.south |- bus.north);
        \draw[<->,thick] (busunit3.north) -- (busunit3.north |- bus.south);
        \draw[<->,thick] (busunit4.north) -- (busunit4.north |- bus.south);
      \end{tikzpicture}
    }
    \subfigure[IBusMaster.busWriteCallback(IBus,Address)]{
      \begin{tikzpicture}
        \node [bus]       (bus)                    {IBus};
        \node [busmaster] (busmaster) at (-4.0, 0) {IBusMaster};
        \node [busunit]   (busunit1)  at (-1.5, 1) {IBusUnit};
        \node [busunit]   (busunit2)  at ( 1.5, 1) {IBusUnit};
        \node [busunit]   (busunit3)  at (-1.5,-1) {IBusUnit};
        \node [busunit]   (busunit4)  at ( 1.5,-1) {IBusUnit};

        \draw[<-,ultra thick,color=red] (busmaster.east) -- (bus.west);
        \draw[<->,thick] (busunit1.south) -- (busunit1.south |- bus.north);
        \draw[<->,thick] (busunit2.south) -- (busunit2.south |- bus.north);
        \draw[<->,thick] (busunit3.north) -- (busunit3.north |- bus.south);
        \draw[<->,thick] (busunit4.north) -- (busunit4.north |- bus.south);
      \end{tikzpicture}
    }
  \end{center}

  \caption{Protokol za sabirnicu (pisanje); slično vrijedi i za
    čitanje samo što se onda podaci prenose preko callback metoda}
  \label{bus_protocol}
\end{figure}


  \newpage

  \section{Word}
  Razred Word omogućuje prikaz riječi proizvoljne širine u bitovima.
  Podržane su logičke operacije, operacije zbrajanja i oduzimanja i
  operacije pomaka.

  Riječi su rastavljene na blokove od 32-bita. Koristi se maksimalno
  24 od 32 bita kako bi se omogućilo byte poravnavanje. Prilikom
  operacija zbrajanja i oduzimanja operacije se vrše blok po blok.
  Operacije zbrajanje i oduzimanja omogućuju unos carry bita kao
  i overflow bit koji izlazi van.

  \tikzstyle{blok}=[draw,rectangle,minimum width=1cm,minimum height=0.5cm]

  \begin{figure}
    \begin{tikzpicture}
      \node[blok] (blok) {};
    \end{tikzpicture}

    \caption{Operacija zbrajanja}
  \end{figure}

  \section{Mikroinstrukcije}
  Mikroinstrukcije predstavljaju različite podatkovne puteve.
  Programski su realizirane preko Command patterna.  Trenutno su
  realizirane sljedeće:
  \begin{description}
    \item[AddMicroinstruction] zbraja 2 registra i rezultat sprema u
      registar za rezultat
    \item[SubtractMicroinstruction] oduzima 2 registra i rezultat u
      registar za rezultat
    \item[ConditionalMicroinstruction] izvršava određenu
      podmikroinstrukciju u slučaju da je zadovoljen neki od uvjeta
    \item[JumpMicroinstruction] skače na određenu lokaciju
    \item[MoveMicroinstruction] vrijednost jednog registra sprema u
      drugi registar
    \item[LoadMicroinstruction] učitava iz memorije (s određene
      adrese) u određeni registar
    \item[StoreMicroinstruction] sprema u memoriju na određenu adresu
      iz određenog registra
  \end{description}

  \todo{Dodati dijagram koji pobliže ilustrira mikroinstrukcije}

  \tikzstyle{executionunit}=[draw, rectangle, minimum width=2cm,
    minimum height=2cm]

  \tikzstyle{registerfile}=[draw, rectangle, minimum width=1cm,
    minimum height=2cm]

  \begin{tikzpicture}
    \node[executionunit] (stage1) at (0,0) {IExecutionUnit};
    \node[registerfile]  (regs1)  at (2,0) {};
    \node[executionunit] (stage2) at (5,0) {IExecutionUnit};
    \node[registerfile]  (regs2)  at (7,0) {};

    \pgfsetcolor{black!20};
    \pgfusepath{fill,stroke};

    \foreach \t in {0,0.25,0.5,0.75,1}{
      \pgfpathmoveto{\pgfpointlineattime{\t}{stage1.north east}{stage1.south east}};
      \pgfpathlineto{\pgfpointlineattime{\t}{regs1.north west}{regs1.south west}};
    }

    \draw[->] (stage1.east) -- (regs1.west);
    \draw[->] (regs1.east) -- (stage2.west);
    \draw[->] (stage2.east) -- (regs2.west);
  \end{tikzpicture}

  \subsection{Poboljšanja}

  Trenutne mikroinstrukcije bi se možda trebale spustiti razinu niže
  tako da uistinu predstavljaju podatkove puteve (prema RTL - register
  transfer level). Eventualno bi se trenutne mikroinstrukcije mogle
  ostaviti kakve jesu, a dodati novi paket koji bi omogućio bolju
  reprentaciju podatkovnog puta (stavljanje vrijednosti u registar,
  aritmetičke operacije nad registrom itd...)

\end{document}
