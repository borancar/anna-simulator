\documentclass[a4paper]{article}
\usepackage[utf8x]{inputenc}
\usepackage[croatian]{babel}
\usepackage{geometry}
\usepackage{tikz}
\usepackage{subfigure}
\usepackage{todonotes}

\usetikzlibrary{shapes}

\begin{document}
  \title{ANNA}
  \maketitle

  \section{Kostur}
  Kostur sačinjavaju sučelja koja se nalaze pod
  hr.anna.interfaces. Sučelja predstavljaju očekivanja i mogućnosti
  svake od komponenata.

  Protokoli koje kostur omogućava su neovisni o implementaciji. Za
  primjer potpunosti uzmimo trojku IBusMaster, IBus i
  IBusUnit. IBusMaster može pristupati jedinicama na sabirnici preko
  sljedećeg protokola:

  IBusMaster poziva IBus.requestWrite i preko nje postavlja zahtjev za
  čitanje. Sabirnica obavlja translaciju adrese i dodatne operacije
  potrebne za pristup i prosljeđuje zahtjev preko
  IBusUnit.requestWrite. Jedinica na sabirnici nakon obavljene
  operacije uzvraća preko IBus.busUnitWriteCallback. Sabirnica opet
  obavlja određene translacije i na kraju dostavlja informaciju preko
  IBusMaster.busWriteCallback. Protokol je ilustriran na slici
  \ref{bus_protocol}.

  \subsection{Kritike i poboljšanja}
  Protokol je većinom prilagođen za komunikacije na sabirnicama gdje
  postoji jedan ili par upravljača (mastera) i više jedinica. Pozitivna
  stvar ovoga je što je većina internih protokola master-slave koncepta,
  a i neki eksterni protokoli. Ova sučelja trenutno ne predviđaju složene
  arbitracije i to je jedno od područja koje se može dodatno razvijati.

  Sučelja također omogućuju bridgeve među sabirnicama. Bridge jedinica
  je za jednu sabirnicu samo IBusUnit, a za drugu IBusMaster.

  \todo{Dodati tablicu koja nudi summary protokola koje je moguće/nije
    moguće implementirati}

  \tikzstyle{busmaster} = [draw=red, fill=blue!20, thick, rectangle,
  rounded corners, inner sep=0.2cm]

\tikzstyle{busunit} = [draw=red, fill=blue!20, thick,
  rectangle, rounded corners, inner sep=0.2cm]

\tikzstyle{bus} = [draw=black, fill=white, thick, rectangle, minimum
  width=5cm, minimum height=1pt]

\begin{figure}[hbt!]
  \begin{center}
    \subfigure[Prije početka komunikacije]{
      \begin{tikzpicture}
        \node [bus]       (bus)                    {IBus};
        \node [busmaster] (busmaster) at (-4.0, 0) {IBusMaster};
        \node [busunit]   (busunit1)  at (-1.5, 1) {IBusUnit};
        \node [busunit]   (busunit2)  at ( 1.5, 1) {IBusUnit};
        \node [busunit]   (busunit3)  at (-1.5,-1) {IBusUnit};
        \node [busunit]   (busunit4)  at ( 1.5,-1) {IBusUnit};

        \draw[<->,thick] (busmaster.east) -- (bus.west);
        \draw[<->,thick] (busunit1.south) -- (busunit1.south |- bus.north);
        \draw[<->,thick] (busunit2.south) -- (busunit2.south |- bus.north);
        \draw[<->,thick] (busunit3.north) -- (busunit3.north |- bus.south);
        \draw[<->,thick] (busunit4.north) -- (busunit4.north |- bus.south);
      \end{tikzpicture}
    }
    \subfigure[IBus.requestBusWrite(IBusMaster,Address,Data)]{
      \begin{tikzpicture}
        \node [bus]       (bus)                    {IBus};
        \node [busmaster] (busmaster) at (-4.0, 0) {IBusMaster};
        \node [busunit]   (busunit1)  at (-1.5, 1) {IBusUnit};
        \node [busunit]   (busunit2)  at ( 1.5, 1) {IBusUnit};
        \node [busunit]   (busunit3)  at (-1.5,-1) {IBusUnit};
        \node [busunit]   (busunit4)  at ( 1.5,-1) {IBusUnit};

        \draw[->,ultra thick,color=red] (busmaster.east) -- (bus.west);
        \draw[<->,thick] (busunit1.south) -- (busunit1.south |- bus.north);
        \draw[<->,thick] (busunit2.south) -- (busunit2.south |- bus.north);
        \draw[<->,thick] (busunit3.north) -- (busunit3.north |- bus.south);
        \draw[<->,thick] (busunit4.north) -- (busunit4.north |- bus.south);
      \end{tikzpicture}
    }
    \subfigure[IBusUnit.requestBusUnitWrite(IBus,TranslatedAddress,Data)]{
      \begin{tikzpicture}
        \node [bus]       (bus)                    {IBus};
        \node [busmaster] (busmaster) at (-4.0, 0) {IBusMaster};
        \node [busunit]   (busunit1)  at (-1.5, 1) {IBusUnit};
        \node [busunit]   (busunit2)  at ( 1.5, 1) {IBusUnit};
        \node [busunit]   (busunit3)  at (-1.5,-1) {IBusUnit};
        \node [busunit]   (busunit4)  at ( 1.5,-1) {IBusUnit};

        \draw[<->,thick] (busmaster.east) -- (bus.west);
        \draw[<-,ultra thick, color=red] (busunit1.south) --
        (busunit1.south |- bus.north);
        \draw[<->,thick] (busunit2.south) -- (busunit2.south |- bus.north);
        \draw[<->,thick] (busunit3.north) -- (busunit3.north |- bus.south);
        \draw[<->,thick] (busunit4.north) -- (busunit4.north |- bus.south);
      \end{tikzpicture}
    }
    \subfigure[IBus.busUnitWriteCallback(IBusUnit,TranslatedAddress)]{
      \begin{tikzpicture}
        \node [bus]       (bus)                    {IBus};
        \node [busmaster] (busmaster) at (-4.0, 0) {IBusMaster};
        \node [busunit]   (busunit1)  at (-1.5, 1) {IBusUnit};
        \node [busunit]   (busunit2)  at ( 1.5, 1) {IBusUnit};
        \node [busunit]   (busunit3)  at (-1.5,-1) {IBusUnit};
        \node [busunit]   (busunit4)  at ( 1.5,-1) {IBusUnit};

        \draw[<->,thick] (busmaster.east) -- (bus.west);
        \draw[->,ultra thick, color=red] (busunit1.south) --
        (busunit1.south |- bus.north);
        \draw[<->,thick] (busunit2.south) -- (busunit2.south |- bus.north);
        \draw[<->,thick] (busunit3.north) -- (busunit3.north |- bus.south);
        \draw[<->,thick] (busunit4.north) -- (busunit4.north |- bus.south);
      \end{tikzpicture}
    }
    \subfigure[IBusMaster.busWriteCallback(IBus,Address)]{
      \begin{tikzpicture}
        \node [bus]       (bus)                    {IBus};
        \node [busmaster] (busmaster) at (-4.0, 0) {IBusMaster};
        \node [busunit]   (busunit1)  at (-1.5, 1) {IBusUnit};
        \node [busunit]   (busunit2)  at ( 1.5, 1) {IBusUnit};
        \node [busunit]   (busunit3)  at (-1.5,-1) {IBusUnit};
        \node [busunit]   (busunit4)  at ( 1.5,-1) {IBusUnit};

        \draw[<-,ultra thick,color=red] (busmaster.east) -- (bus.west);
        \draw[<->,thick] (busunit1.south) -- (busunit1.south |- bus.north);
        \draw[<->,thick] (busunit2.south) -- (busunit2.south |- bus.north);
        \draw[<->,thick] (busunit3.north) -- (busunit3.north |- bus.south);
        \draw[<->,thick] (busunit4.north) -- (busunit4.north |- bus.south);
      \end{tikzpicture}
    }
  \end{center}

  \caption{Protokol za sabirnicu (pisanje); slično vrijedi i za
    čitanje samo što se onda podaci prenose preko callback metoda}
  \label{bus_protocol}
\end{figure}


  \newpage

  \section{Word}
  Razred Word omogućuje prikaz riječi proizvoljne širine u bitovima.
  Podržane su logičke operacije, operacije zbrajanja i oduzimanja i
  operacije pomaka.

  Riječi su rastavljene na blokove od 32-bita. Koristi se maksimalno
  24 od 32 bita kako bi se omogućilo byte poravnavanje. Prilikom
  operacija zbrajanja i oduzimanja operacije se vrše blok po blok.
  Operacije zbrajanje i oduzimanja omogućuju unos carry bita kao
  i overflow bit koji izlazi van.

  \tikzstyle{blok}=[draw,rectangle,minimum width=1cm,minimum height=0.5cm]

  \begin{figure}
    \begin{tikzpicture}
      \node[blok] (blok) {};
    \end{tikzpicture}

    \caption{Operacija zbrajanja}
  \end{figure}

  \section{Mikroinstrukcije}
  Zadatak dekodera instrukcija je generacija kontrolnih signala unutar
  procesora. Jedan ovakav općeniti dekoder se može razbiti na 2 manja
  dekodera:
  \begin{enumerate}
    \item Dekoder instrukcije u jednu ili više instrukcija (nazovimo
      ih mikroinstrukcije)
    \item Dekoder mikroinstrukcija u kontrolne signale unutar procesora
  \end{enumerate}

  Povijesni razlog uvođenja mikroinstrukcija je bio olakšavanje
  dekodiranja kompleksnih instrukcija. Mikroinstrukcije (mikrokod
  općenito) su olakšali rješavanje sljedećih problema kod CISC
  procesora:
  \begin{itemize}
    \item Kako je set instrukcija postajao sve kompleksniji, sve je
      teže bilo pronaći efikasno kodiranje kojim bi se olakšala
      aktivacija podatkovnih putova unutar procesora
    \item Paralelizacija koju je moguće napraviti često nije bila
      očita tokom dizajna procesora
    \item Kompleksna povezanost između seta instrukcija i podatkovnih
      putova
    \item Reprogramabilnost u slučaju pronađenog kvara (Intel Core 2)
  \end{itemize}

  Za RISC i VLIW procesore postoje argumenti za i protiv uporabe
  mikroinstrukcija, međutim, mi ih koristimo radi jednostavnosti i
  dosljednosti arhitekture.  Najčešće će za RISC procesore vrijediti
  preslikavanje 1 na 1. VLIW procesori kodiraju više jednostavnih
  operacija unutar jedne instrukcije i stoga ovdje opet očekujemo
  preslikavanje 1 na 1 za svaku operaciju.

  Mikroasembler (IInstructionDecoder u našem slučaju) rastavlja ulaznu
  instrukciju na mikroinstrukcije koje mikromašina unutar procesora
  izvršava. U HW-u je to realizirano na način da postoji
  mikrosekvencer koji iterira po mikroinstrukcjima i izvršava
  ih. Mikroasembler odgovara prvom dekoderu, dok mikrosekvencer predaje
  rezultat prvog dekodera 2. dekoderu. Mikrosekvencer i 2. dekoder se
  mogu grupirati u mikromašinu.

  Ovisno o tome da li je potrebno naknadno dekodiranje prije nego
  se kontrolni signali generiraju, postoje horizontalne i vertikalne
  mikroinstrukcije. Vraćajući se na početno razbijanje na 2 dekodera,
  horizontalne mikroinstrukcije imaju trivijalni oblik 2. dekodera.  U
  programskoj realizaciji nije moguće napraviti ovakvu podjelu
  (programska realizacija već sama po sebi omogućava i koristi
  prekompleksno dekodiranje) pa je stoga i nećemo.

  Očekivane prednosti jednake su prednostima koje VM-ovi
  trenutno nude jer su mikroinstrukcije tandem IL, a mikromašina
  tandem VM. \todo{Popisati ih, LLVM kao najbliža referenca}
  Kao što VM i IL bytecode nude portabilnost, tako i
  mikroinstrukcije apstrahirane ovdje trebaju nuditi presliku bilo
  kojeg instrukcijskog seta u jedan predefinirani set naredbi koje će
  naše ponuđene mikromašine znati izvršavati. Pri tome se jedna
  instrukcija iz seta preslikava u jednu ili više
  mikroinstrukcija.

  Programski su realizirane preko Command patterna.  Trenutno su
  realizirane sljedeće:
  \begin{description}
    \item[AddMicroinstruction] zbraja 2 registra i rezultat sprema u
      registar za rezultat
    \item[SubtractMicroinstruction] oduzima 2 registra i rezultat u
      registar za rezultat
    \item[ConditionalMicroinstruction] izvršava određenu
      podmikroinstrukciju u slučaju da je zadovoljen neki od uvjeta
    \item[JumpMicroinstruction] skače na određenu lokaciju
    \item[MoveMicroinstruction] vrijednost jednog registra sprema u
      drugi registar
    \item[LoadMicroinstruction] učitava iz memorije (s određene
      adrese) u određeni registar
    \item[StoreMicroinstruction] sprema u memoriju na određenu adresu
      iz određenog registra
  \end{description}

  \subsection{Poboljšanja}

  Treba izdvojiti općenitu implementaciju mikromašine (nazovimo je
  IMicroMachine ili IExecutionUnit) tako da je dovoljno samo je
  uključiti u procesor za omogućiti rad. Tada je također moguće i
  specijalizirati mikromašinu da izvršava samo određene
  mikroinstrukcije (podskup), a ostale prosljeđuje. Nastavak na ovo su
  razne kombinacije grafove koje će predstavljati
  serijske/paralelne/kombinirane podatkovne putove.

  Za dodati pipeline treba dodati i posebnu instrukciju za flush
  pipeline kada se dogodi hazard koji nije moguće obraditi. Nakon
  toga, dovoljno je samo pospojiti specificirane mikromašine u seriju
  stupnjeva koliko ih pipeline ima.

  Može se dodati poseban IDispatcher čija je služba raspodjeljivanje posla
  kada postoji više paralelnih mikromašina koje mogu izvršavati isti podskup
  mikroinstrukcija.

  Trenutne mikroinstrukcije bi se možda trebale spustiti razinu niže
  tako da uistinu predstavljaju podatkovne puteve (prema RTL -
  register transfer level). Primjerice, JumpMicroinstruction bi se
  mogla razbiti na MoveMicroinstruction odrediša u PC i
  LoadMicroinstruction sa adrese PC u IR. Eventualno bi se trenutne
  mikroinstrukcije mogle ostaviti kakve jesu (jer služe svrsi
  savršeno), a dodati novi paket koji bi omogućio bolju reprentaciju
  podatkovnog puta (stavljanje vrijednosti u registar, aritmetičke
  operacije nad registrom itd...).

\end{document}
